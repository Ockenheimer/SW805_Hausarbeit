%!TEX root = ../hausarbeit.tex
%!TEX spellcheck = de_DE
% Hauptmenuepunkt
\section{Einleitung}\label{Einleitung}
%\addcontentsline{toc}{section}{Einleitung}
Die Lebensgefährtin des Autors ist Leitung der Kinderkrippe Tannenweg in Ingelheim. Da der Autor Student der Sozialinformatik an der Hochschule Fulda ist, trat die Leitung an Ihn heran, um das bislang genutzte Übergabeheft in eine digitale Form zu überführen und den Prozess somit weiter zu verbessern.

In dieser Hausarbeit soll nun der ursprüngliche Prozess nach QM-Standards analysiert werden. Anschließend sollen Verbesserungsmöglichkeiten und die Einsatzmöglichkeiten der Sozialinformatik aufgezeigt werden.

\section{Das Unternehmen}

Das behandelte Unternehmen ist die Kinderkrippe "{}Tannenweg"{} in Ingelheim. Die Einrichtung ist in kommunaler Trägerschaft, es besteht jedoch ein Kooperationsvertrag mit dem lokalen Pharmaunternehmen Boehringer Ingelheim. Das Haus verfügt über vier Gruppen mit insgesamt 40 Betreuungsplätzen. In den vier Gruppen arbeiten insgesamt sieben pädagogische Fachkräfte und fünf Kinderkrankenschwestern so wie zwei hauswirtschaftliche Kräfte.

Die Einrichtung ist täglich von 6:45 Uhr bis 17:30 Uhr bzw. Freitags bis 14:00 Uhr geöffnet. Folglich besteht ein Schichtsystem der Vollzeitkräfte, um die langen Öffnungszeiten abzudecken.

\subsection{Der Prozess}

Die Weitergabe von Informationen zwischen den Schichtsystemen ist ein zentrales Thema in der Kinderkrippe.
Zur Abstimmung der Informationen und zur Dokumentation von Vorkommnissen im Betreuungsalltag wird in den einzelnen Gruppen ein Übergabeheft geführt. In diesem Heft werden kinder-betreffende Informationen über den Tag dokumentiert. Ist das Kind z.B. während der Frühschicht gefallen, wird dies im Übergabeheft vermerkt, sodass der Spätdienst diese Information bei der Übergabe des Kindes an die Eltern mitteilen kann.

Die Einführung des Heftes wurde durch eine damalige Gruppenkollegin durch eine Projektarbeit im Rahmen Ihres Studiums begleitet. Es zeigte sich eine enorme Verbesserung der Qualität im pädagogischen Arbeitsalltag der Einrichtung.

Das ursprünglich eingeführte Übergabeheft war ein für die Gruppe angeschafftes DinA5-Heft, in dem die entsprechenden Informationen vermerkt wurden. Nach der Verbreitung des Konzeptes auf die anderen Gruppen wurde der Wunsch nach einer Vereinheitlichung und einer Vereinfachung des Konzeptes laut.

Informationen werden zur Zeit im Laufe des Tages auf einem laminierten Papier mit wasserlöslichem Folienschreiber vermerkt. Der Spätdienst reinigt jeden Abend zum Dienstende das Papier, damit der Frühdienst morgens keine Zeit mit Vorbereitungsarbeiten verliert.

Leider läuft der Prozess noch nicht optimal. 
Informationen die der Spätdienst gesammelt hat, werden durch die Reinigung nicht für den Frühdienst des nächsten Tages konserviert. 

\section{QM in Kindertageseinrichtungen}

\blindtext


\section{Optimierung des Prozesses}

\subsection{Prozessanalyse}
\blindtext

\subsection{Ansatzpunkte der Sozialinformatik}
\textquote{[Komplexe Technologien]...beeinflussen immer die sozialen Systeme, in denen sie eingesetzt werden und werden umgekehrt wiederum von von diesen Systemen adaptiert und beeinflusst}\citep[][24]{kreidenweis2012}

\section{Fazit}

\blindtext